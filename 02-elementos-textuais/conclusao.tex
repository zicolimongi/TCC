\chapter{Conclusão}
\label{chap:conclusao}

A persistência poliglota pode melhorar o desempenho da aplicação, mas o modelo deve ser bem estudado, para que haja uma melhora no sistema como um todo. Os resultados demonstram que a persistência poliglota melhorou parte do sistema, mas não o sistema como um todo.

Para o funcionamento do Twitter o modelo utilizado não foi ideal, pois o tempo de inserção de um \textit{tweet} aumentou muito. Porém, é um modelo que funcionaria muito bem em aplicações que há uma grande quantidade de leitura e uma pequena quantidade de escrita.

Este trabalho apresentado é apenas um começo do estudo de persistência poliglota. Podemos fazer diferentes análises, como a quantidade de espaço utilizado, consistência das informações e outras para afirmar qual modelo é melhor. A próxima etapa seria aprofundar os testes realizados nas duas aplicações e em seguida colocar ambas em um ambiente paralelo.


