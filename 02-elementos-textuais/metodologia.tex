%
% Documento: Metodologia
%

\chapter{Metodologia}

Este trabalho será realizado em três etapas.

A primeira etapa consiste do estudo dos bancos de dados escolhidos, buscando entender melhor como os dados são estruturados. Ainda nessa etapa, será feito uma pesquisa bibliográfica sobre sistemas que utilizam persistência poliglota. Essa pesquisa será fundamental para a próxima etapa, pois irá direcionar a criação da aplicação para que ressalte as diferenças de um sistema com persistência poliglota.

A segunda etapa consiste da criação de duas aplicações semelhantes ao Twitter, porém uma utilizará apenas um banco de dados e a outra utilizará dois bancos de dados. Essa aplicação terá as funcionalidades básicas do Twitter.

A terceira e última etapa é a comparação da performance das operações sobre os dados dessas aplicações. Logo, consiste na medição de tempo gasto para inserção, leitura, atualização e exclusão dos dados.

Para a realização desse trabalho será necessário de um computador com os servidores dos bancos de dados instalados.

\section{Implementação}
\label{sec:implementation}
Explicar a aplicação, como foi feito e apresentar os casos de uso

\section{Implementação Monoglota}
\label{sec:monoglot}
Apresentar os diagrama de classe da parte persistida e da aplicação como um todo, falar do framework MongoId, explicar a forma com que a associcao dos models

\section{Implementação Poliglota}
\label{sec:polyglot}
Apresentar os diagrama de classe da parte persistida e da aplicação como um todo, falar do framework do Redis, explicar onde foi usado um banco e onde foi usado outro.

\section{Coleta de dados}
\label{sec:data}
Explicar como foi coletado os dados
