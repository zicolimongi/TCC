\chapter{Resultados Preliminares}
\label{chap:resultados}

Para demostrar os resultados preliminares utilizamos a \textit{gem} rack-mini-profiler. Essa \textit{gem} mede o tempo total que a aplicação leva para renderizar a página. Sabemos que o alvo deste trabalho é o tempo gasto nas consultas do banco de dados, mas essa ferramenta auxiliou para confirmarmos o melhor desempenho da implementação com persistência poliglota.

Os teste foram realizados na máquina Asus, modelo N82J\footnote{A especificação se encontra no sítio da Asus \url{http://www.asus.com/Notebooks_Ultrabooks/N82Jq/}}, porém nessa márquina foi adicionado mais 4GB de RAM e o sistema operacional é Ubuntu 14.04 LTS.

A modelo de persistência poliglota tinha como alvo melhorar o desempenho da página \verb|feed|, mas sabemos que para isso a escrita do \textit|tweet| iria ser mais lenta. Para fazer um teste, no qual esses tempos medidos resultassem em uma diferença significativa populamos o banco de dados MongoDB com mil usuários e cada usuário com cem \textit{tweets}, totalizando em um total de cem mil \textit{tweets}.

Executamos as açõe \verb|feed| e cadastro de \textit{tweet} cem vezes para cada aplicação.

A média de resultados encontrados pode ser visualizado na \autoref{tab:tempo}
\begin{table}[H]
    \centering
    \caption[Média de tempo das aplicações em milisegundos]{Média de tempo das aplicações em milisegundos.\label{tab:tempo}}
    \begin{tabular}{ccc}
        \hline
            Ação & Persistência Monoglota & Persistência Poliglota\\
        \hline
            \verb|feed| & 280,76666 & 24,66666 \\
            Cadastro Tweet & 176,62333 & 308,85534 \\
        \hline
    \end{tabular}
    \fonte{Autoria própria.}
\end{table}


Como esperado a página \verb|feed| teve um grande ganho de performance utilizando a persistência poliglota, mas o cadastro de \textit{tweet} ficou mais lento. Esses resultados são simples, apenas para provar que pode haver melhora utilizando persistência poliglota, mas uma análise mais profunda é necessária inclusive aumentando o volume de dados para visualizar a reação do sistema.