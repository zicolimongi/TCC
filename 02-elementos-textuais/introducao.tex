\chapter{Introdução}\label{chap:introducao}
A necessidade de persistência de dados sempre esteve presente na computação. A medida que os sistemas evoluíram a complexidade da forma que os dados eram armazenados aumentou significativamente. Com isso houve a necessidade da criação de um sistema computadorizado de manutenção de registros \cite{CJDate}, o banco de dados.

O modelo relacional foi um dos primeiros gêneros de banco de dados, sua estrutura são tabelas de duas dimensões com linhas e colunas. Os dados armazenados são tipados podendo variar a quantidade de tipos de acordo com o banco utilizado. Para interagir com esse gênero de banco é necessário realizar consultas com a linguagem SQL. Alguns exemplos de banco de dados relacional são MySQL, SQLite, Oracle e PostgreSQL.

Durante anos, o banco de dados relacional tinha sido considerado a melhor opção para os problemas de pequena e grande escalabilidade, porém surgiram novas soluções com novas alternativas de estruturas, replicação simples, alta disponibilidade e novos métodos de consultas \cite{SDSW}. Essas opções são conhecidas como NoSQL ou banco de dados não-relacional

Existem diversos gêneros de banco de dados não-relacional, entre eles chave-valor, orientado à documento, orientado à coluna e orientado à nó. Com o surgimento dessas novas soluções, o questionamento sobre qual banco de dados é melhor para resolver certo tipo de problema, veio à tona. A partir disso, o conhecimento e compreensão sobre os bancos de dados em geral se torna necessário para realizar uma boa escolha.

Entendendo que cada banco se destaca em determinados tipos de problema, é nítido perceber que sistemas que trabalham com mais de um banco de dados podem oferecer um melhor desempenho, dando origem à persistência poliglota.

Este trabalho consiste na comparação de dois sistemas, o primeiro um que utilizará apenas um banco de dados e o segundo que utilizará dois bancos de dados ou persistência poliglota. Os bancos escolhidos para fazer esses sistemas foram o MongoDB e o Redis. O autor escolheu esses bancos de dados por ter experiência na linguagem \textit{Ruby on Rails} que oferece excelentes bibliotecas para esses bancos. O intuito desse trabalho é comprovar que o uso da persistência poliglota melhora o desempenho da aplicação.

MongoDB é um banco de dados do gênero orientado à documento e foi desenhado para ser gigante, como o próprio nome é uma derivação da palavra inglesa \textit{humongous} que significa gigantesco. Além disso, tem soluções interessantes para evitar leitura suja e as consultas são realizadas na linguagem JavaScript. A diferença desse genêro é que são armazenados documentos compostos de um identificador único e um conjunto de valores de tipos e estruturas aninhadas, chamadas de BSON, uma estrutura parecida com JSON. Esse gênero é bem flexível, pois não tem \textit{schema}, ou seja, não existe tabelas. A organização dos dados é feito por documentos, ao criar a arquitetura do sistema temos que identificar se as entidades criadas são expressivas como um documento \cite{SDSW}. Nesse trabalho iremos utilizar o banco MongoDB nos dois sistemas que serão criados. O MongoDB está sendo utilizado em grandes empresas, como Cisco, eBay, Codeacademy, Microsoft, New York Time, Craiglist, The Guardian e outras referencia site mongo??.

O segundo banco que iremos utilizar se chama Remote Dictionary Server ou Redis do gênero chave-valor. Esse tipo de armazenamento é mais simples, como próprio nome indica, é armazenado um valor para determinada chave. Essa escolha foi feita devido ao cache que esse sistema realiza antes de efetivar a operação no disco. Esse cache tem um ganho muito alto em performance, porém poderá ocorrer perda de dados, caso ocorra uma falha de hardware \cite{SDSW}. A forma de como estruturar esse banco é muito parecida com um tipo estruturado chamado \textit{hash} que são implementadas em algumas linguagens de computação, como Java e Ruby. Esse banco será utilizado no segundo sistema a ser desenvolvido. O Redis está sendo utilizado no Twitter, Github, Craiglist e outros referencia site REDIS.


\section{Motivação}
\label{sec:motivacao}
A persistência poliglota é uma alternativa para melhorar o desempenho de uma aplicação. Utilizando-a conseguimos adaptar cada tipo de problema com um gênero de banco.  Em uma mesma aplicação podemos ter um conjunto dados que devem estar sempre disponíveis e em outra parte da aplicação um conjunto de dados que devem ser consistentes. Então podemos separar os dados dessa aplicação em bancos com gêneros diferentes, cada um irá armazenar os dados, no qual o seu gênero se destaca.








