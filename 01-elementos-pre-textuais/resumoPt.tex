%
% Documento: Resumo (Português)
%

\begin{resumo}

Persistência poliglota é a utilização de mais de um gênero de banco de dados por uma mesma aplicação. Em um sistema podemos ter algum conjunto de dados que deve ter alta disponibilidade, e outro conjunto de dados que deve ser consistente. Devido a essa diferença de necessidades surgiu a persistência poliglota. Utilizando-a podemos armazenar em um banco, que garante disponibilidade, o conjunto de dados que precisa de alta disponibilidade e podemos armazenar em outro banco, que favorece a consistência, o conjunto de dados que precisa ser consistente. Visando esse problema, criamos duas aplicações semelhantes ao Twitter, sendo uma com a persistência monoglota, que utiliza apenas o banco de dados MongoDB e a outra com persistência poliglota que utiliza os bancos de dados MongoDB e Redis. Fizemos a comparação entre esses dois sistemas sobre duas funcionalidades, a leitura do \textit{feed} de \textit{tweets} e a inserção de um \textit{tweet}. Com o modelo criado utilizando persistência poliglota, conseguimos melhorar o tempo de leitura do \textit{feed} de \textit{tweets}, porém o tempo de inserção do \textit{tweet} piorou. Percebemos que a persistência poliglota possibilitou uma melhora no desempenho em parte da aplicação. Por isso, podemos afirmar que o modelo escolhido não foi ideal para ressaltar as vantagens de um sistema que utiliza persistência poliglota, mas demonstramos que a utilização da persistência poliglota pode melhorar muito o desempenho da aplicação.


\textbf{Palavras-chave}: Banco de dados. Persistência poliglota. NoSQL. Desempenho. 

\end{resumo}