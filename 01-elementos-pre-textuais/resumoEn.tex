%
% Documento: Resumo (Inglês)
%

\begin{resumo}[Abstract]


Polyglot persistence is a system that use more than one database. In a system can have a set of data that must be available, and another set of data that must be consistent. The need to maintain these two types of data, did rise the polyglot persistence. Thus, the set of data, that needs to be available are stored in a database, that favors availability, and the set of data, that needs to be consistent are stored in a database, that favors consistency.We created two applications similar to Twitter, one using just MongoDB database, and another using Redis and MongoDB databases. We did tests to compare the spent in reading the tweet feed and the time spent to insert a tweet. With the polyglot persistence the time spent in reading the feed of tweets was lower and the time spent to insert was higher. Therefore, the model choosen was not the ideal for the Twitter similar system, but we proved that polyglot persistence can improve the performance of the application.

\textbf{Keywords}: Database. Polyglot Persistence. NoSQL. Performance. 

\end{resumo}
